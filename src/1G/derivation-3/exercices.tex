%\documentclass[12pt]{article}
\documentclass[twocolumn,10pt]{article}
\usepackage[a4paper,landscape]{geometry}

\usepackage[utf8]{inputenc}
\usepackage[frenchb]{babel}


\usepackage{amsthm}
\usepackage{amsmath}
\usepackage{amsfonts}
\usepackage{amssymb}
\usepackage{graphicx}
\usepackage{mathtools}
\usepackage{enumitem}
\usepackage{fancybox}
\usepackage{geometry}
\usepackage{calrsfs}
\usepackage{tikz}
\usepackage{tkz-tab}
\usepackage{eurosym}
\usepackage{comment}
\usepackage{tabularx}

\theoremstyle{definition}
\newtheorem{theo}{Théoreme}
\theoremstyle{definition}
\newtheorem{propriete}[theo]{Propriété}
\newtheorem{algorithme}[theo]{Algorithme}
\newtheorem{axiom}[theo]{Axiom}
\newtheorem{case}[theo]{Case}
\newtheorem{claim}[theo]{Claim}
\newtheorem{conclusion}[theo]{Conclusion}
\newtheorem{condition}[theo]{Condition}
\newtheorem{conjecture}[theo]{Conjecture}
\theoremstyle{definition}
\newtheorem{coro}[theo]{Corollaire}
\newtheorem{criterion}[theo]{Criterion}
\theoremstyle{definition}
\newtheorem{definition}[theo]{Définition}
\theoremstyle{definition}
\newtheorem{exemple}[theo]{Exemple}
\theoremstyle{definition}
\newtheorem{exemples}[theo]{Exemples}
\theoremstyle{definition}
\newtheorem{exercice}{Exercice}
\theoremstyle{definition}
\newtheorem{correction}{Correction}
\newtheorem{lemme}[theo]{Lemme}
\theoremstyle{definition}
\newtheorem{notation}[theo]{Notation}
\newtheorem{probleme}[theo]{Problème}
\theoremstyle{definition}
\newtheorem{proposition}[theo]{Proposition}
\theoremstyle{definition}
\newtheorem{remarque}[theo]{Remarque}
\theoremstyle{definition}
\newtheorem{remarques}[theo]{Remarques}
\newtheorem{solution}[theo]{Solution}
\theoremstyle{definition}
\newtheorem{vocabulaire}[theo]{Vocacbulaire}
\theoremstyle{definition}
\newtheorem{illu}[theo]{Illustration}
\theoremstyle{definition}
\newtheorem{pb}[theo]{Problème}
\theoremstyle{definition}
\newtheorem{cpart}[theo]{Cas particulier}
\theoremstyle{definition}
\newtheorem{cours}{Question de cours}
\theoremstyle{definition}
\newtheorem{rappel}[theo]{Rappel}
\newenvironment{methode}[1][Méthode]{\textbf{#1.} }{\ \rule{0.5em}{0.5em}}
\newenvironment{dem}[1][Démonstration]{\textbf{#1.} }{\ \rule{0.5em}{0.5em}}

\renewcommand{\thecours}{\empty{}} 
\renewcommand{\thecorrection}{\empty{}} 

\renewcommand\FrenchLabelItem{\textbullet} 

\newcommand{\V}{\overrightarrow}
\newcommand{\ps}[2]{\V{#1}\cdot\V{#2}}
\newcommand{\N}[1]{\left\|\V{#1}\right\|}
\newcommand{\va}[1]{\left|#1\right|}
\newcommand{\Det}[4]{\begin{vmatrix} #1 & #3 \\ #2 & #4 \end{vmatrix}}
\newcommand{\limite}[2]{\lim\limits{#1\rightarrow#2}}

\geometry{hmargin=1.5cm,vmargin=1.5cm}

\setlength{\columnseprule}{0.5pt}
\setlength{\columnsep}{30pt}

\newcommand\coord[2]{\begin{pmatrix}
 #1 \\
 #2 
 \end{pmatrix}}

\renewcommand{\arraystretch}{2.5} 

\begin{document}

\begin{center}
\Large\textbf{Dérivation et composées}\normalsize
\end{center}

%%%%%%%%%%%%%%%%%%%%%%%%%%%%%%%%%%%%%%%%%%%%%%%%%%%%%%%%%%%%%%%%%%%%%%%%%%%%%%%%%%%%%%%%%%%%%%

\begin{exercice}~\\
Pour chacune des fonctions suivantes, calculer la fonction dérivée.
\begin{center}
\begin{tabularx}{12cm}{XX}
$1.~~f_1(x)=(-3x+8)^7$ &$2.~~f_2(x)=e^{2x+3}$\\ 
$3.~~f_3(x)=10e^{-5x+5}$ &$4.~~f_4(x)=\sqrt{5x-7}$ \\
$5.~~f_5(x)=(1+\sqrt{x})^5$ &$6.~~f_6(x)=e^{\sqrt{3x+1}}$\\
$7.~~f_7(x)=\dfrac{1}{\sqrt{x}}$ (par 2 façons différentes)\\
\end{tabularx}
\end{center}
\end{exercice}

\begin{exercice}~\\
On éteint le chauffage dans une pièce d'habitation à 22h. La température y est alors de $20$\degre C. \\
Le but de cet exercice est d'étudier l'évolution de la température de la pièce entre 22h et 7h le lendemain matin. \\
On suppose que la température extérieur est constante et égale à $11$\degre C. \\
On désigne par $t$ le temps écoulé depuis 22h, exprimé en heure, et par $f(t)$ la température de la pièce exprimée en \degre C. La température est donc modélisée par une fonction $f$ définie sur $[0;9]$. 
\begin{enumerate}
\item Prévoir le sens de variation de la fonction $f$ sur $[0;9]$. 
\item On admet désormais que la fonction $f$ est définie par : $$f(t)=9e^{-0,12t}+11$$
			\begin{enumerate} 
			\item Donner une justification mathématique au sens de variation trouvé dans la question 1.
			\item Calculer une valeur approchée, à $10^{-2}$ près, de $f(9)$ et interpréter ce résultat. 
			\item À l'aide de la calculatrice, représenter la fonction $f$ et déterminer l'heure à partir de laquelle la température est inférieur à $15$\degre C. 
			\end{enumerate}
\end{enumerate}
\end{exercice}



\begin{exercice}~\\
Pour un individu $A$, on enregistre la fréquence cardiaque pendant la phase de récupération après un test d'effort de 8 minutes. \\
On admet que cette fréquence peut être modélisée par la fonction $g$ définie sur $[8;13]$ par : $$g(t)=660e^{-0,163t}$$
où le temps $t$ est donné en minutes (min) et $g(t)$ en battements par minute.
\begin{enumerate}
\item On appelle $\mathcal{C}$ la courbe représentative de $g$. \\
Justifier que la fonction $g$ est décroissante. 
\item À l'aide de la calculatrice, tracer l'allure de la courbe $\mathcal{C}$. 
\item En déduire, par lecture graphique, le temps de récupération, exprimé en minutes et seconde, à partir duquel la fréquence cardiaque est inférieur ou égale à $115$ battements par minutes. 
\item L'étude de l'évolution de la fréquence cardiaque après un test d'effort donne des renseignements sur le profil cardio-vasculaire d'un individu. \\
Ainsi, une diminution de la fréquence cardiaque inférieur à $12$ battements lors de la première minute est considérée comme anormale et peut indiquer un problème d'ordre médial. \\
La fréquence cardiaque de récupération de l'individu $A$ peut-elle être considérée comme anormale ?
\end{enumerate}   
\end{exercice}

\begin{exercice}~\\
Soit $f$ et $g$ deux fonctions définies sur $\mathbb{R}$ par : $$f(x)=\dfrac{e^x+e^{-x}}{2}~~~~\text{ et }~~~~g(x)=\dfrac{e^x-e^{-x}}{2}$$
\begin{enumerate}
\item Montrer que pour tout réel $x$ on a $f'(x)=g(x)$ et $g'(x)=f(x)$. 
\item En déduire le tableau de variation de la fonction $g$. 
\item Résoudre l'équation $g(x)=0$. 
\item Dresser le tableau de variation de la fonction $f$. 
\item Montrer que pour tout réel $x$ on a : $f(x)^2-g(x)^2=1$
\end{enumerate}
\end{exercice}

\begin{exercice}~\\
Soit $f$ la fonction définie sur $[2;+\infty[$ par : $$f(x)=x^2\sqrt{2x-4}$$
\begin{enumerate}
\item Reconnaître les fonctions $u$ et $v$ pour que $f$ soit de la forme $u\times v$. 
\item déterminer leurs ensemble de dérivabilité puis calculer les fonction $u'(x)$ et $v'(x)$. 
\item En déduire l'ensemble de dérivabilité de $f$. 
\item Montrer que $f'(x)=\dfrac{5x^2-8x}{\sqrt{2x-4}}$ pour tout $x$ de l'ensemble de dérivabilité. 
\item En déduire le tableau de variation de $f$ sur $[2;+\infty[$.
\end{enumerate}
\end{exercice}







\end{document}